\chapter{Tanulásmenedzsment rendszerek}
Tanulásmenedzsment rendszernek, angolul Learning Management Systemnek (LMS-nek) nevezzük azokat a szoftver alkalmazásokat, amelyek automatizálják az oktatás adminisztrációját, követését, az online kurzusok és az azokkal kapcsolatos események, anyagok kezelését.

Egy robusztus LMS-nek képesnek kell lennie \cite{link:ell}:
\begin{sajat_itemize}
\setlength{\itemsep}{0pt}
\item központosított és automatizált adminisztrációra,
\item önkiszolgáló és önálló irányítású szolgáltatások nyújtására,
\item oktatási anyagok gyors összeállítására és elérhetőségének biztosítására,
\item konszolidált képzési kezdeményezésekre skálázható, web alapú platformon,
\item a portabilitás és a szabványok támogatására,
\item személyre szabott tartalom előállítására és a tudás újrafelhasználásának lehetővé tételére.
\end{sajat_itemize}

Tehát egy tanulásmenedzsment rendszer nem más, mint olyan szolgáltatások összessége, amelyek támogatják a rendszer felhasználóinak adminisztrálását, jogosultságok kiosztását, új tartalmak, oktatási anyagok létrehozását, elérhetővé tételét, megosztását, a felhasználók tanulmányi menetének követését, irányítását, felkészültségüknek ellenőrzését, általában webes platformon, más rendszerekkel szabványosan együtt működve. Az LMS-ek általában lehetőséget biztosítanak, arra is, hogy a felhasználók a rendszeren keresztül kommunikáljanak is egymással.

Fontos megjegyezni, hogy nem csak a közoktatásban, hanem a vállalati, ipari továbbképzésekben is szívesen alkalmazzák ezeket a rendszerek, mivel sok közülük rendelkezik a megfelelő humánerőforrás kezeléssel kapcsolatos funkciókkal is.

A piacon több nyílt forrású, ingyenesen elérhető és zárt, kereskedelmi változatokat, vagy vegyes koncepciókat is találunk, amelyek esetében ugyan a rendszer maga nyílt forrású, de a támogatásért, vagy akár bérüzemeltetésért már pénzbeli juttatást kérnek. \Aref{tab:openlms}.~táblázatban néhány ismertebb nyílt forrású LMS-t soroltam fel \cite{link:lms}. Érdemes megnézni, hogy általában PHP, MySQL technológiák segítségével valósítják meg ezeket a rendszereket.

\definecolor{MyTableColor}{rgb}{0.38,0.28,0.25} 

\begin{table}[h]
	\caption{Néhány elterjedtebb LMS}
	\centering
	\small
	\begin{tabular}{| p{1.6cm} | p{4.4cm} | p{2.2cm} | p{3.8cm} |}
		\hline
		\rowcolor{MyTableColor} \textbf{Név} & \textbf{Projekt oldal} & \textbf{Használt prog.~nyelv} & \textbf{Támogatott adatbázisok} \\
		\hline
		Sakai & \href{http://www.sakaiproject.org/}{http://www.sakaiproject.org/} & Java & MySQL,~Oracle,~DB2 \\
		\hline
		Moodle & \href{http://moodle.org/}{http://moodle.org/} & PHP & MySQL, PostgreSQL, MSSQL, Oracle \\
		\hline
		OLAT &  \href{http://www.olat.org/}{http://www.olat.org/} & Java & MySQL,~PostgreSQL \\
		\hline
		Instructure - Canvas & \href{http://www.instructure.com/}{http://www.instructure.com/} & Ruby on Rails & PostgreSQL \\
		\hline
		ILIAS & \href{http://www.ilias.de/docu/}{http://www.ilias.de/docu/} & PHP & MySQL, Oracle 11g \\
		\hline
		Dokeos & \href{http://www.dokeos.com/}{http://www.dokeos.com/} & PHP & MySQL \\
		\hline
		Chamilo & \href{http://www.chamilo.org/}{http://www.chamilo.org/} & PHP & MySQL \\
		\hline		
		Claroline & \href{http://www.claroline.net/}{http://www.claroline.net/} & PHP & MySQL \\
		\hline
	\end{tabular}
	\normalsize
	\label{tab:openlms}
\end{table}


\todo{A táblázat még mindig gyönyörű. Janem. \# Low priority}

\todo{Adott a kérdés, hogy mi lehet ennek az oka? \# Érdemes vele foglalkozni?} 

Kereskedelmi termékekről nem sikerült érdemleges információhoz jutni az alkalmazott technológiákról, de feltételezhető a Java, .NET platformok használata is. A felhasználói oldalon gyakran találunk Flash alapú megvalósításokat.

A HTML5 technológia terjedésének valószínűsíthető következménye lesz, hogy ezekben a rendszerekben is lecserélésre kerül a Flash-es keretrendszer. A HTML5 lehetőséget biztosít arra, hogy az eddigi megjelenést ne kelljen lecserélni, s ezzel a felhasználónak ne kelljen az új rendszerbe beletanulni, ugyanakkor hasonló felülettel találkozzanak az otthoni számítógépük, netbookjuk, táblagépük, vagy mobiltelefonjuk használata során. Ezek mellett fejlesztési oldalon költségcsökkenésre lehet számítani, mert nem kell a különböző eszközökre külön-külön lefejleszteni ugyanazt a grafikus interfészt, ami a HTML5 szabványosságának köszönhető. Az üzemeltetésre is kevesebb teher jut, hiszen ezen technológia nagy része felhasználói oldalon fut, így nem csökken a rendszer teljesítménye.

\todo{Miről lehet még itt írni?}

\todo{Bibliográfia}

\begin{comment}

% TODO

\bibitem[ell] {link:ell}
	Ellis, Ryann K. {\it A Field Guide to Learning Management Systems}, ASTD Learning Circuits, 2009 \\ \href{http://www.astd.org/NR/rdonlyres/12ECDB99-3B91-403E-9B15-7E597444645D/23395/LMS\_fieldguide\_20091.pdf}{http://www.astd.org/NR/rdonlyres/.../LMS\_fieldguide\_20091.pdf}

\bibitem[lms] {link:lms}
	Wikipedia, {\it List of learning management systems} \\ \href{http://en.wikipedia.org/wiki/List\_of\_learning\_management\_systems}{http://en.wikipedia.org/wiki/List\_of\_learning\_management\_systems}

\end{comment}
