\chapter{Tanulásmenedzsment rendszerek}

Ebben a fejezetben definiálom a tanulásmenedzsment vagy oktatástámogató rendszerek fogalmát, bemutatom lényegesebb tulajdonságaikat, és néhány példát mutatok a piacon elérhető megvalósításaikról. A fejezet végén az ezen rendszerek közötti átjárhatóság megvalósítását részletezném.

\section{A tanulásmenedzsment rendszerek fogalma}
Tanulásmenedzsment rendszernek, angolul Learning Management Systemnek (LMS-nek), vagy oktatástámogató rendszernek nevezzük azt a szoftver alkalmazást, amely automatizálja az oktatás adminisztrációját, követését, az online kurzusok és az azokkal kapcsolatos események, anyagok kezelését.

Egy robusztus LMS-nek képesnek kell lennie \cite{ellis2009}:
\begin{sajat_itemize}
\item központosított és automatizált adminisztrációra,
\item önkiszolgáló és önálló irányítású szolgáltatások nyújtására,
\item oktatási anyagok gyors összeállítására és elérhetőségének biztosítására,
\item konszolidált képzési kezdeményezésekre skálázható, web alapú platformon,
\item a portabilitás és a szabványok támogatására,
\item személyre szabott tartalom előállítására és a tudás újrafelhasználásának lehetővé tételére.
\end{sajat_itemize}

Tehát egy tanulásmenedzsment rendszer nem más, mint olyan szolgáltatások összessége, amelyek támogatják a rendszer felhasználóinak adminisztrálását, jogosultságok kiosztását, új tartalmak, oktatási anyagok létrehozását, elérhetővé tételét, megosztását, a felhasználók tanulmányi menetének követését, irányítását, felkészültségük ellenőrzését, mindezt általában webes platformon, más rendszerekkel szabványosan együtt működve. Az LMS-ek általában lehetőséget biztosítanak arra is, hogy a felhasználók a rendszeren keresztül kommunikáljanak is egymással.

Fontos megjegyezni, hogy nem csak a közoktatásban, hanem a vállalati, ipari továbbképzésekben is szívesen alkalmazzák ezeket a rendszerek, mivel sok közülük rendelkezik a megfelelő humánerőforrás-kezeléssel kapcsolatos funkciókkal is. Például a Hewlett-Packard egyik alkalmazottjától származó információk szerint a cégnél felhőalapú oktatási rendszert használnak.

\section{A piacon elérhető oktatástámogató rendszerek}

A piacon különböző nyílt forrású, ingyenesen elérhető és zárt, kereskedelmi változatokat vagy vegyes koncepciókat is találunk, amelyek esetében ugyan a rendszer maga nyílt forrású, de a támogatásért vagy akár bérüzemeltetésért már pénzbeli juttatást kérnek. \Aref{tab:openlms}.~táblázatban néhány ismertebb nyílt forrású LMS-t soroltam fel \cite{lms}. Érdemes megnézni, hogy általában PHP, MySQL technológiák segítségével valósítják meg ezeket a rendszereket.

\definecolor{MyTableColor}{rgb}{0.38,0.28,0.25} 

\begin{table}[h]
	\caption{Néhány elterjedtebb LMS}
	\centering
	\small
	\begin{tabular}{| p{1.6cm} | p{4.4cm} | p{2.2cm} | p{3.8cm} |}
		\hline
		\rowcolor{MyTableColor} \textbf{Név} & \textbf{Projekt oldal} & \textbf{Használt prog.~nyelv} & \textbf{Támogatott adatbázisok} \\
		\hline
		Sakai & \href{http://www.sakaiproject.org/}{http://www.sakaiproject.org/} & Java & MySQL,~Oracle,~DB2 \\
		\hline
		Moodle & \href{http://moodle.org/}{http://moodle.org/} & PHP & MySQL, PostgreSQL, MSSQL, Oracle \\
		\hline
		OLAT &  \href{http://www.olat.org/}{http://www.olat.org/} & Java & MySQL,~PostgreSQL \\
		\hline
		Instructure - Canvas & \href{http://www.instructure.com/}{http://www.instructure.com/} & Ruby on Rails & PostgreSQL \\
		\hline
		ILIAS & \href{http://www.ilias.de/docu/}{http://www.ilias.de/docu/} & PHP & MySQL, Oracle 11g \\
		\hline
		Dokeos & \href{http://www.dokeos.com/}{http://www.dokeos.com/} & PHP & MySQL \\
		\hline
		Chamilo & \href{http://www.chamilo.org/}{http://www.chamilo.org/} & PHP & MySQL \\
		\hline		
		Claroline & \href{http://www.claroline.net/}{http://www.claroline.net/} & PHP & MySQL \\
		\hline
	\end{tabular}
	\normalsize
	\label{tab:openlms}
\end{table}


Kereskedelmi termékekről nem sikerült érdemleges információhoz jutni az alkalmazott technológiákról, de feltételezhető a Java, .NET platformok használata is. A felhasználói oldalon gyakran találunk Flash alapú megvalósításokat.

A HTML5 technológia terjedésének valószínűsíthető következménye lesz, hogy ezekben a rendszerekben is lecserélésre kerül a Flash-es keretrendszer. A HTML5 lehetőséget biztosít arra, hogy az eddigi megjelenést ne kelljen lecserélni, s ezzel a felhasználónak ne kelljen az új rendszerbe beletanulni, ugyanakkor hasonló felülettel találkozzanak az otthoni számítógépük, netbookjuk, táblagépük vagy mobiltelefonjuk használata során. Ezek mellett fejlesztési oldalon költségcsökkenésre lehet számítani, mert nem kell a különböző eszközökre külön-külön lefejleszteni ugyanazt a grafikus interfészt, ami a HTML5 szabványosságának köszönhető. Az üzemeltetésre is kevesebb teher jut, hiszen ezen technológia nagy része felhasználói oldalon fut, így nem csökken a rendszer teljesítménye.

Egyre több olyan céget találunk az Interneten, amelyek valamilyen oktatástámogató megoldást nyújtanak. Ezek egy része megvásárolható szoftveralkalmazás, amelyet a felhasználónak kell telepítenie és üzemeltetnie, de egyre jobban elterjednek a felhő alapú megoldások, amelyek esetében a cég csak az alkalmazás használatát adja bérbe, az üzemeltetésével ő foglalkozik, levéve ezzel a felhasználó válláról a terhet.

\section{Átjárhatóság az egyes LMS-ek között}

Jogosan merül fel a kérdés, hogy ha ilyen sok LMS érhető el a piacon, melyik a számunkra megfelelő választás, és mi van abban az esetben, ha később rájövünk arra, hogy teljesen más igényeket kell kielégítenünk, ezért szeretnénk egy másik alkalmazást használni. Vagy csak egyszerűen szeretnék a már meglévő tudástárunkat egyszerűen migrálni egy másik oktatástámogató rendszerbe. Ez a probléma már korábban is megjelent, megoldására találták ki a SCORM-ot.

\section{SCORM}
A SCORM (Sharable Content Object Reference Model) web-alapú e-learning rendszerekkel  kapcsolatos szabványok és specifikációk gyűjteménye. A SCORM definiálja a kommunikációt a kliens oldal és a futtatási környezet (run-time environment) között \footnote{\href{http://scorm.com/scorm-explained/}{http://scorm.com/scorm-explained/}}.

Felhasználói oldalról nézve a tanárok oktatási csomagokat tölthetnek fel/le a kurzusok között akár különböző LMS megvalósítások esetén is, ha azok támogatják az adott SCORM verziót (portabilitás).

A SCORM megalkotása az ADL (Advanced Distributed Learning) Initiative\footnote{\href{http://www.adlnet.org/}{http://www.adlnet.org/}} nevű amerikai kormányzati szervezethez kötődik. Az ADL-t 1997-ben alapították a Védelmi minisztérium felügyelete alatt, feladata oktatás technológiai alkalmazások fejlesztése és implementálása. Később több országban nyitottak kutatólaboratóriumot partnereiknél.

Az interoperabilitás infrastruktúrája a SCORM-on és az ADL-R-en (ADL Registry-n) alapul. A SCORM felelős az LMS-ek között a kurzusok és az azokhoz kapcsolódó adatok cseréjéért, a kurzusok közötti tartalom újrafelhasználásért és a tanulóra szabott tartalom sorrendjéért. Az ADL-R a különböző tartalomkönyvtárakban és média tárolókban való keresést segíti elő.

A SCORM előnyei \cite{sco}:
\begin{sajat_itemize}
\item objektumalapú megközelítést biztosít az oktatási anyag fejlesztéshez és szállításához
\item interoperabilitást biztosít ezekhez az objektumokhoz többfajta szállítási környezeten keresztül
\item a tanuló saját ritmusú haladásán alapuló szofisztikált tanulási stratégiákat enged meg
\item lehetővé teszi tananyagok és oktatási stratégiák importálását, exportálását.
\end{sajat_itemize}
