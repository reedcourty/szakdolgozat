\chapter{Tanulásmenedzsment rendszerek}
LMS-nek (Learning Management System) nevezzük azokat a szoftver alkalmazásokat, amelyek automatizálják az oktatás adminisztrációját, követését, az online kurzusok és az azokkal kapcsolatos események, anyagok kezelését.

Egy robusztus LMS-nek képesnek kell lennie \cite{link:ell}:
\begin{itemize}
\item központosított és automatizált adminisztrációra
\item önkiszolgáló és önálló irányítású szolgáltatások nyújtására
\item oktatási anyagok gyors összeállítására és szállítására
\item konszolidált képzési kezdeményezésekre skálázható, web alapú platformon
\item a portabilitás és a szabványok támogatására
\item személyre szabott tartalom előállítására és a tudás újrafelhasználásának lehetővé tételére.
\end{itemize}

\Aref{tab:openlms}.~táblázatban néhány ismertebb nyílt forrású LMS-t soroltam fel \cite{link:lms}. Érdemes megnézni, hogy általában PHP, MySQL technológiák segítségével valósítják meg ezeket a rendszereket.

\definecolor{MyTableColor}{rgb}{0.38,0.28,0.25} 

\begin{table}[h]
	\caption{Néhány elterjedtebb LMS}
	\centering
	\small
	\begin{tabular}{| p{1.6cm} | p{4.4cm} | p{2.2cm} | p{3.8cm} |}
		\hline
		\rowcolor{MyTableColor} \textbf{Név} & \textbf{Projekt oldal} & \textbf{Használt prog.~nyelv} & \textbf{Támogatott adatbázisok} \\
		\hline
		Sakai & \href{http://www.sakaiproject.org/}{http://www.sakaiproject.org/} & Java & MySQL,~Oracle,~DB2 \\
		\hline
		Moodle & \href{http://moodle.org/}{http://moodle.org/} & PHP & MySQL, PostgreSQL, MSSQL, Oracle \\
		\hline
		OLAT &  \href{http://www.olat.org/}{http://www.olat.org/} & Java & MySQL,~PostgreSQL \\
		\hline
		Instructure - Canvas & \href{http://www.instructure.com/}{http://www.instructure.com/} & Ruby on Rails & PostgreSQL \\
		\hline
		ILIAS & \href{http://www.ilias.de/docu/}{http://www.ilias.de/docu/} & PHP & MySQL, Oracle 11g \\
		\hline
		Dokeos & \href{http://www.dokeos.com/}{http://www.dokeos.com/} & PHP & MySQL \\
		\hline
		Chamilo & \href{http://www.chamilo.org/}{http://www.chamilo.org/} & PHP & MySQL \\
		\hline		
		Claroline & \href{http://www.claroline.net/}{http://www.claroline.net/} & PHP & MySQL \\
		\hline
	\end{tabular}
	\normalsize
	\label{tab:openlms}
\end{table}

