\begin{table}[h]
	\caption{Néhány ismertebb PaaS szolgáltató}
	\centering
	\small
	\begin{tabular}{| p{4cm} | p{5.5cm} | p{4cm} |}
		\hline
		\rowcolor{MyTableColor} \textbf{Szolgáltató} & \textbf{URL} & \textbf{Platform} \\
		\hline
		Google AppEngine & \href{https://appengine.google.com}{https://appengine.google.com} & Python, Java, Go \\ 
		\hline
		Heroku & \href{http://www.heroku.com/}{http://www.heroku.com/} & Ruby, Node.js, Clojure, Java, Python, Scala \\
		\hline
		Epio & \href{https://www.ep.io/}{https://www.ep.io/} & Python (Django, Pylons, Pyramid, Flask, Trac) \\
		\hline
		Zend PHP Cloud Application Platform\footnote{Nem igazi szolgáltatás, csak egy platform, amelyet pl. Amazon EC2-re lehet telepíteni.} & \href{http://www.zend.com/en/products/php-cloud/}{http://www.zend.com/en/products/php-cloud/} & PHP\\
		\hline
		SpringSource\footnote{Szintén egy telepíthető platform, VMware vFabric Cloud Application Platform alapokon.} & \href{http://www.springsource.com/}{http://www.springsource.com/} & Java (Groovy, Grails)\\
		\hline
	\end{tabular}
	\normalsize
	\label{tab:paas_providers}
\end{table}

\todo{footnote-okat kiszedni szövegbe!}