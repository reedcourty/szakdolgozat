\chapter{Tanulásmenedzsment rendszerek erőforrásigényei}
\section{Alapvető igények}
\subsection{Az operációs rendszer igényei}
Minden informatikai rendszernek van erőforrásigénye, vagyis az a minimális hardver konfiguráció, amelyen a rendszer bizonyos számú kéréseket ki tud szolgálni adott időn belül. \todo{Kellene jobb fogalom leírás}
Egy informatikai rendszer teljes erőforrásigénye sok részletből tevődhet össze. Alapvetően meghatározza az operációs rendszer igénye, az erre épülő szolgáltatások (adatbázis szerver, webszerver, levelezőszerver, egyebek) és a főszolgáltatás platformja.
A ma használatos szerver operációs rendszerek alapkövetelményének számít:

\todo{Forrás: \href{http://msdn.microsoft.com/en-us/windowsserver/cc196364}{link}}
\todo{Forrás: \href{https://help.ubuntu.com/11.10/installation-guide/i386/minimum-hardware-reqts.html}{link}}
\todo{Forrás: \href{https://www.redhat.com/rhel/compare/}{link}}

\begin{itemize}
\item min. 1-1,5 GHz-es CPU (x86 vagy x64), de 2 GHz az ajánlott,
\item min. 512 MB memória, de 2 GB ajánlott \footnote{Red Hat Linux esetében pl. 1 GB/processzor egység},
\item min. 10 GB tárkapacitás.
\end{itemize}

\subsection{Eltérő platformok, eltérő igények}
Nem csak az LMS-ek, de minden webes alkalmazás esetén is különböző platformokat találunk az alkalmazás szerver oldalán. Az igen elterjedt, középkategóriás PHP mellett az üzleti életben gyakran Java alapokon fejlesztenek, de nem szabad megfeledkezni a .NET-es, Ruby on rails-es, vagy a Python-os megvalósításokról sem.
Az egyes platformok különböznek teljesítményükben, ''gyorsaságukban'', megbízhatóságukban, de a ,,Melyik nyelven fejlesszük le ezt a rendszert?'' kérdés esetén nem feltétlenül kapjuk meg azonnal a helyes választ.
\todo{miért jó a php?} 

\todo{Webszerverek eltérő igénye}
\todo{Platform igények PHP vs. Java vs. Ruby vs. ASP vs. Python}
\todo{adatbázisok igénye MySQL vs. PostgreSQL vs. Oracle vs. MSSQL}

\subsection{Moodle}
A Moodle hardveres erőforrásigényei a következők:
\begin{itemize}
\item min. 160 MB tárolókapacitás (csak a rendszernek),
\item min. 256 MB memóriakapacitás (1 GB ajánlott).
\end{itemize}
\todo{CPU}
\todo{I/O}
\todo{Forrás: \href{http://docs.moodle.org/20/en/Installing\_Moodle}}


\section{Igények terhelésváltozások esetén}
\todo{}
\section{Modellek}
\todo{}
