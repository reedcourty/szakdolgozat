\chapter{Tanulásmenedzsment rendszerek erőforrásigényei}
\section{Alapvető igények}
\subsection{Az operációs rendszer igényei}
Minden informatikai rendszernek van erőforrásigénye, vagyis az a minimális hardver konfiguráció, amelyen a rendszer bizonyos számú kéréseket ki tud szolgálni adott időn belül. \todo{Kellene jobb fogalom leírás}
Egy informatikai rendszer teljes erőforrásigénye sok részletből tevődhet össze. Alapvetően meghatározza az operációs rendszer igénye, az erre épülő szolgáltatások (adatbázis szerver, webszerver, levelezőszerver, egyebek) és a főszolgáltatás platformja.
A ma használatos szerver operációs rendszerek alapkövetelményének számít \cite{ws2008sr}, \cite{ubuntuminhr}, \cite{redhatcaplim}:

\begin{sajat_itemize}
\item min. 1-1,5 GHz-es CPU (x86 vagy x64), de 2 GHz az ajánlott,
\item min. 512 MB memória, de 2 GB ajánlott \footnote{Red Hat Linux esetében pl. 1 GB/processzor egység},
\item min. 10 GB tárkapacitás.
\end{sajat_itemize}

Gyakran ajánlják a szerverek operációs rendszerének a Linux rendszereket, mivel ezek köztudottan kevesebb memóriát használnak fel, így több marad a többi szolgáltatás számára. 

\subsection{Eltérő platformok, eltérő igények}
Nem csak az LMS-ek, de minden webes alkalmazás esetén is különböző platformokat találunk az alkalmazás szerver oldalán. Az igen elterjedt, középkategóriás PHP mellett az üzleti életben gyakran Java alapokon fejlesztenek, de nem szabad megfeledkezni a .NET-es, Ruby on rails-es, vagy a Python-os megvalósításokról sem. Az egyes platformok különböznek teljesítményükben, ''gyorsaságukban'', megbízhatóságukban és erőforrásigényükben is.

Az egyik legelterjedtebb webes alkalmazás platform a PHP, mivel ez a nyelv szinten minden operációs rendszeren, szinte minden webszerverrel együtt tud működni, és emellett igen jól skálázható.

\subsection{Webszerverek erőforrásigénye}

\todo{Webszerverek eltérő igénye}

\subsection{Adatbázisszerverek erőforrásigénye}

\todo{adatbázisok igénye MySQL vs. PostgreSQL vs. Oracle vs. MSSQL}

\section{Egy példarendszer: a Moodle}
A Moodle hardveres erőforrásigényei a következők:
\begin{sajat_itemize}
\item min. 160 MB tárolókapacitás (csak a rendszernek \footnote{A teljes tárolókapacitás függ az oktatási anyag mennyiségétől, méretétől, típusától (pl. oktatóvideók)}),
\item min. 256 MB memóriakapacitás (1 GB ajánlott).
\end{sajat_itemize}
Érdemes megjegyezni, hogy a hivatalos Moodle oldalon 1 GB memóriát ajánlanak 50 felhasználó egyszerre történő kiszolgálásához.
Természetesen ez még nem a teljes rendszer igénye, hiszen a webkiszolgálónak és az adatbázisszervernek is van erőforrásigénye. 
\todo{CPU}
\todo{I/O}
\todo{Forrás: \href{http://docs.moodle.org/20/en/Installing\_Moodle}}

\section{Igények terhelésváltozások esetén}
\todo{}
\section{Modellek}
\todo{}
