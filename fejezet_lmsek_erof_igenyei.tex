\chapter{Tanulásmenedzsment rendszerek erőforrásigényei}
\section{Alapvető igények}
\todo{•}

Minden informatikai rendszernek van erőforrásigénye, vagyis az a minimális hardver konfiguráció, amelyen a rendszer bizonyos számú kéréseket ki tud szolgálni adott időn belül. \todo{Kellene jobb fogalom leírás}
A rendszer erőforrásigénye sok részletből tevődhet össze. 
\todo{Milyen OS-en vagyunk, annak is van alap igénye}
\todo{Webszerverek eltérő igénye}
\todo{Platform igények PHP vs. Java vs. Ruby vs. ASP vs. Python}
\todo{adatbázisok igénye MySQL vs. PostgreSQL vs. Oracle vs. MSSQL}

\subsection{Moodle}
A Moodle hardveres erőforrásigényei a következők:
\begin{itemize}
\item{min. 160 MB tárolókapacitás (csak a rendszernek),}
\item{min. 256 MB memóriakapacitás (1 GB ajánlott).}
\end{itemize}
\todo{CPU}
\todo{I/O}
\todo{Forrás: \href{http://docs.moodle.org/20/en/Installing\_Moodle}}


\section{Igények terhelésváltozások esetén}
\todo{}
\section{Modellek}
\todo{}
