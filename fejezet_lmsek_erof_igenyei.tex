\chapter{Tanulásmenedzsment rendszerek erőforrásigényei}
\section{Alapvető igények}
\subsection{Az operációs rendszer igényei}
Minden informatikai rendszernek van erőforrásigénye, vagyis az a minimális hardver konfiguráció, amelyen a rendszer bizonyos számú kéréseket ki tud szolgálni adott időn belül.

\todo{Kellene jobb fogalom leírás}

Egy informatikai rendszer teljes erőforrásigénye sok részletből tevődhet össze. Alapvetően meghatározza az operációs rendszer igénye, az erre épülő szolgáltatások (adatbázis szerver, webszerver, levelezőszerver, egyebek) és a főszolgáltatás platformja.
A ma használatos szerver operációs rendszerek alapkövetelményének számít \cite{ws2008sr,ubuntuminhr,redhatcaplim}:

\begin{sajat_itemize}
\item min. 1-1,5 GHz-es CPU (x86 vagy x64), de 2 GHz az ajánlott,
\item min. 512 MB memória, de 2 GB ajánlott \footnote{Red Hat Linux esetében pl. 1 GB/processzor egység},
\item min. 10 GB tárkapacitás.
\end{sajat_itemize}

Gyakran ajánlják a szerverek operációs rendszerének a Linux rendszereket, mivel ezek köztudottan kevesebb memóriát használnak fel, így több marad a többi szolgáltatás számára. 

\subsection{Eltérő platformok, eltérő igények}
Nem csak az LMS-ek, de minden webes alkalmazás esetén is különböző platformokat találunk az alkalmazás szerver oldalán. Az igen elterjedt, középkategóriás PHP mellett az üzleti életben gyakran Java alapokon fejlesztenek, de nem szabad megfeledkezni a .NET-es, Ruby on rails-es, vagy a Python-os megvalósításokról sem. Az egyes platformok különböznek teljesítményükben, ''gyorsaságukban'', megbízhatóságukban és erőforrásigényükben is.

Az egyik legelterjedtebb webes alkalmazás platform a PHP, mivel ez a nyelv szinte minden operációs rendszeren, szinte minden webszerverrel együtt tud működni, és emellett igen jól skálázható.

\subsection{Webszerverek erőforrásigénye}

\todo{Webszerverek eltérő igénye}

\subsection{Adatbázisszerverek erőforrásigénye}

\todo{adatbázisok igénye MySQL vs. PostgreSQL vs. Oracle vs. MSSQL}

\section{Egy példarendszer: a Moodle}
A Moodle hardveres erőforrásigényei a következők:
\begin{sajat_itemize}
\item min. 160 MB tárolókapacitás (csak a rendszernek \footnote{A teljes tárolókapacitás függ az oktatási anyag mennyiségétől, méretétől, típusától (pl. oktatóvideók)}),
\item min. 256 MB memóriakapacitás (1 GB ajánlott).
\end{sajat_itemize}
Érdemes megjegyezni, hogy a hivatalos Moodle oldalon 1 GB memóriát ajánlanak 50 felhasználó egyszerre történő kiszolgálásához.
Természetesen ez még nem a teljes rendszer igénye, hiszen a webkiszolgálónak és az adatbázisszervernek is van erőforrásigénye \cite{moodleinst}.
 
\todo{CPU}

\todo{I/O}

\section{Igények terhelésváltozások esetén}

\subsection{Terhelésváltozások okozói LMS-eknél}

Az LMS-ek esetében a terhelés változások okozói elsősorban maguk a felhasználók, hiszen ők azok, akik időszakosan megnövekedett kérésszámot intézhetnek a rendszerünk felé. Érdemes lehet ezeket az időszakokat és a hozzájuk kapcsolódó viselkedési változásokat összegyűjteni.

Egy LMS életében a következő erőforrás igényes időszakokat különböztethetjük meg:
\begin{sajat_itemize}
\item kurzus-/vizsgajelentkezési időszak,
\item kurzussal kapcsolatos feladatok beadási határideje,
\item kurzus online teszt, vagy vizsga kitöltés (határ)ideje,
\item egyéb a kurzussal kapcsolatos offline számonkérés,
\item online előadás közvetítés,
\item audiovizuális tananyagokkal rendelkező kurzus számonkérésének ideje, 
\end{sajat_itemize}
és még hosszasan sorolhatnánk.

A következőekben megpróbálom összefoglalni, hogy ezek közül melyik milyen módon terheli meg a rendszerünk erőforrásai.

\subsubsection{Kurzus-/vizsgajelentkezési időszak}

Sajnos a BME hallgatóinak nem ismeretlen a Neptun Egységes Tanulmányi Rendszer vizsga- és tárgyjelentkezési időszakokban történő kiesése. Ezekben az időszakokban rohamosan megnő az egyszerre aktív felhasználók száma, amivel a rendszer nem tud megbirkózni. Az üzemeltetés mindent megtesz annak érdekében, hogy a rendszer működőképes legyen a több ezer felhasználó esetén is, de valamelyik rétegben egy-egy erőforrás nem tudja feldolgozni a kéréseket, ami az egész rendszerre kihatással van.

\cite{neptun1,neptun2,neptun3,neptun4,neptun5}

\todo{Neptunos írást befejezni!}

\subsubsection{Kurzussal kapcsolatos feladatok beadási határideje}

Hallgatók jellemző viselkedése, hogy megpróbálják az utolsó pillanatban elkészíteni a kiadott feladatokat, amelynek következtében a határidő előtti rövid időintervallumban megnövekszik a rendszerben egyszerre tartózkodó felhasználók száma, és állományok feltöltésével esetlegesen túlterhelik a rendszer hálózatát, adatbázisát és nem előrelátó üzemeltetés esetén a tárhely szabad helyének méretében is kritikus szintet érhetnek el.

\todo{Még valami?}

\todo{Melyik részben jelenik meg, hogy hogyan lehet proaktívan kivédeni?}

\subsubsection{Kurzus online teszt, vagy vizsga kitöltés (határ)ideje}

Egy oktatási rendszerben meglehet határozni olyan időintervallumokat, amelyek egy kurzus tesztjének kitöltésére engedélyezett időszak. A tesztet az intervallum elejétől a végéig lehetséges kitölteni, azon kívül a teszt kitöltése letiltott funkció. Ezzel a felhasználók arra vannak kényszerítve, hogy az adott intervallumon belül bejelentkezzenek a rendszerbe, és kitöltsék a megadott tesztet.

Mivel az időintervallum meghatározott (az üzemeltetés is könnyen tudomást szerezhet róla), és a várható felhasználói létszám is ismert (az adott kurzuson résztvevő felhasználók száma a maximum), azért viszonylag jól fel lehet készülni erre az erőforrással kapcsolatos igényváltozásra. Természetesen itt sem tudjuk 100\%-osan meghatározni, hogy melyik időpillanatban hány felhasználó lesz a rendszerben, és a felhasználó viselkedéstől is nagyban függ az eloszlás, de maximális értékeket meghatározhatunk. Persze nem biztos, hogy ez költség szempontjából az optimális megoldás.

\subsubsection{Egyéb a kurzussal kapcsolatos offline számonkérés}

Napjainkban egyre több oktatási anyag kerül fel online tárhelyre, amelyet az LMS-ek alapból támogatnak. Egy-egy számonkérés előtt megnövekedhet ezen oktatási anyagok letöltésének száma, amely az anyag méretétől, számától és a kurzusra jelentkezett felhasználók számától függően terhelheti az adatbázist és a hálózatot.

Ebben az esetben a feltöltésekhez hasonlóan nem lehetséges biztos modellt alkotni az erődforrásigények változásáról, hiszen itt is nagyban függünk a felhasználói viselkedéstől. Míg folyamatosan készülő diákok az egész félévben elosztott terhelést jelentenek, addig az utolsó pillanatosok a számonkérés előtt nem sokkal terhelhetik a rendszert. 

\subsubsection{Online előadás közvetítés}

Az Internet térnyerésével, és a széles sávú kapcsolatok terjedésével megjelent az \angolul{online stream} lehetősége. Egy ilyen közvetítés egy bizonyos időintervallumban terhelheti a hálózatunkat. Az elérhetőségének korlátozásával (pl. csak az adott kurzust felvett felhasználók nézhetik) a maximális nézőszámot is meg tudjuk előre határozni. Tehát összességében egy ilyen eseményre viszonylag jól fel tudjuk készíteni a rendszerünket erőforrások szempontjából.  

\subsubsection{Audiovizuális tananyagokkal rendelkező kurzus számonkérésének ideje}

Ez az eset az online közvetítés és a számonkérések egyvelege. Mivel a számonkérés ideje rögzített, így addig az időpontig számíthatunk arra, hogy a rendszerünket az adott kurzus felhasználói terhelni fogják. A terhelés főleg a hálózatot fogja érinteni.

Nagyon eltérő terheléseket kaphatunk attól függően, hogy az oktatási anyag letölthető-e vagy csak online nézhető. Az előbbi esetben nagyobb az esély az időben jobban szétkenődő erőforrásigény növekedésekre, míg az utóbbinál a határidő előtt nagy konkurens felhasználói számmal és igénynövekedéssel kell számolnunk. Ennek oka lehet, hogy az online elérhető anyagot a számonkérés előtt nem sokkal fogják megnézni a felhasználók, amivel a rendszerünket terhelik. Ellenben a letölthető anyag esetében meg van az a lehetőség, hogy néhány felhasználó már jóval a határidő előtt letölti azt, így csökkentve a maximális felhasználók számát a kritikus időpontban.

\todo{Milyen események lehetnek, amik igényváltozással járnak?}

\todo{Melyik lehet ezek közül problémás?}
\section{Modellek}
\todo{JMeter, logokból esetleg, nagyjából lehet csak mappelni}

\todo{Borsival wadon erőforrásokról beszélni}
