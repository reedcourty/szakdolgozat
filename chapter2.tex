%----------------------------------------------------------------------------
\chapter{A dolgozat formai kivitele}
%----------------------------------------------------------------------------
Az itt található információk egy része a BME VIK Hallgatói Képviselet által készített ,,Utolsó félév a villanykaron'' c. munkából lett kis változtatásokkal átemelve. Az eredeti dokumentum az alábbi linken érhető el: \url{http://vik-hk.bme.hu/diplomafelev-howto-2009}.

%----------------------------------------------------------------------------
\section{A dolgozat kimérete}
%----------------------------------------------------------------------------
A minimális 50, az optimális kiméret 60-70 oldal (függelékkel együtt). A bírálók és a záróvizsga bizottság sem szereti kifejezetten a túl hosszú dolgozatokat, így a bruttó 90 oldalt már nem érdemes túlszárnyalni. Egyébként függetlenül a dolgozat kiméretétől, ha a dolgozat nem érdekfeszítő, akkor az olvasó már az elején a végét fogja várni. Érdemes zárt, önmagában is érthető művet alkotni.

%----------------------------------------------------------------------------
\section{A dolgozat nyelve}
%----------------------------------------------------------------------------
Mivel Magyarországon a hivatalos nyelv a magyar, ezért alapértelmezésben magyarul kell megírni a dolgozatot. Aki külföldi posztgraduális képzésben akar részt venni, nemzetközi szintű tudományos kutatást szeretne végezni, vagy multinacionális cégnél akar elhelyezkedni, annak célszerű angolul megírnia diplomadolgozatát. Mielőtt a hallgató az angol nyelvű verzió mellett dönt, erősen ajánlott mérlegelni, hogy ez mennyi többletmunkát fog a hallgatónak jelenteni fogalmazás és nyelvhelyesség terén, valamint - nem utolsó sorban - hogy ez mennyi többletmunkát fog jelenteni a konzulens illetve bíráló számára. Egy nehezen olvasható, netalán érthetetlen szöveg teher minden játékos számára.

%----------------------------------------------------------------------------
\section{A dokumentum nyomdatechnikai kivitele}
%----------------------------------------------------------------------------
A dolgozatot A4-es fehér lapra nyomtatva, 2,5 centiméteres margóval (+1~cm kötésbeni), 11-12 pontos betűmérettel, talpas betűtípussal és másfeles sorközzel célszerű elkészíteni.


