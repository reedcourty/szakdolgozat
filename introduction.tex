%----------------------------------------------------------------------------
\chapter*{Bevezető}\addcontentsline{toc}{chapter}{Bevezető}
%----------------------------------------------------------------------------

Napjainkban egyre több területen jelenik meg az elektronikus oktatás. Manapság már nem csak a közoktatásban, de az iparban is szívesen alkalmazzák a számítógép és Internet által támogatott tanulást, továbbképzést. Ezzel együtt növekedik az igény az ilyen rendszerek megfelelő, magas rendelkezésre állást biztosító üzemeltetése iránt.

Az ilyen alkalmazások erőforrásigényeik speciális tulajdonságokkal rendelkeznek változásaik szempontjából. Egyrészről ezek az ingadozások determinisztikusak, nagy pontossággal előre meghatározhatóak, másrészről viszont az üzemeltetés szempontjából igényt tartanak az állandó megfigyelésre, analizálásra és időbeli reakciókra.

Az oktatási rendszerek nagyon gyorsan terjednek, ugyanakkor még nincs érdemi múltjuk, ezen felül olyan új technológiákat alkalmaznak, mint pl. a felhőalapú infrastruktúrák, ezért ezen a területen még nem rendelkezünk megfelelő információkkal arról, hogy hogyan tudjuk az erőforrás kihasználtsági problémákat effektíven kezelni. Ezeket a problémákat, és lehetséges megoldásaikat szeretném megmutatni írásomban.

Szakdolgozatomban először szeretném bemutatni az ún. tanulásmenedzsment, vagy oktatás támogató rendszereket, azok főbb tulajdonságait. Kitérnék a piacon található kész rendszerekre, azok megvalósítási platformjaira.

A második részben bemutatnám ezen rendszerek üzemeltetési IT infrastruktúráját, kiindulva az ismert három rétegű architektúra alapján, majd egy példa rendszer felépítését.

A harmadik fejezetben az oktatás támogató rendszerek erőforrásigényeit, azok változásait, és a változások lehetséges okait fogom megvizsgálni. Ugyanebben a részben megpróbálom megmutatni ezen igényváltozások modellezési lehetőségeit.

A következő fejezetben a klasszikus IT infrastruktúra megbízhatóságának, rendelkezésre állás növelésének lehetőségeit járom körül, majd bemutatom az ún. felhő alapú megoldásokat, és azok jellemzőit, szolgáltatásait.

Az utolsó részben a rendszer lehetséges menedzselési típusait mutatom be, részletesebben megvizsgálva a proaktív menedzsment jellemzőit, lehetséges alkalmazásait az oktatás támogató rendszerek üzemeltetése esetén.
 