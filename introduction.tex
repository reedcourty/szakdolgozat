%----------------------------------------------------------------------------
\chapter*{Bevezető}\addcontentsline{toc}{chapter}{Bevezető}
%----------------------------------------------------------------------------

Napjainkban egyre több területen jelenik meg az elektronikus oktatás. Manapság már nem csak a közoktatásban, de az iparban is szívesen alkalmazzák a számítógép és Internet által támogatott tanulást, továbbképzést. Ezzel együtt növekszik az igény az ilyen rendszerek megfelelő, nagy rendelkezésre állást biztosító üzemeltetése iránt.

Az elektronikus oktatástámogató alkalmazások erőforrásigényének változása speciális tulajdonságokkal rendelkezik. Egyrészről ezen ingadozások egy része determinisztikus, nagy pontossággal előre meghatározható, másrészről viszont az üzemeltetés oldaláról igényt tartanak az állandó megfigyelésre, analizálásra és időbeni reakciókra.

Az oktatási rendszerek nagyon gyorsan terjednek, ugyanakkor még nincs érdemi múltjuk, ezen felül olyan új technológiákat alkalmazhatnak, mint pl. a felhőalapú infrastruktúrák, ezért ezen a területen még nem rendelkezünk megfelelő információkkal arról, hogy hogyan tudjuk az erőforrás-kihasználtsági problémákat effektíven kezelni. Ezeket a problémákat és lehetséges megoldásaikat szeretném megmutatni írásomban.

Szakdolgozatomban először szeretném bemutatni az ún. tanulásmenedzsment vagy oktatástámogató rendszereket, azok főbb tulajdonságait. Kitérek a piacon található kész rendszerekre, azok megvalósítási platformjaira.

A második részben az ismert három rétegű architektúra példáján kiindulva bemutatom ezen rendszerek üzemeltetési IT infrastruktúráját, majd egy példa rendszer felépítését.

A harmadik fejezetben az oktatástámogató rendszerek erőforrásigényeit, azok változásait és a változások lehetséges okait vizsgálom. Ugyanebben a részben megmutatom ezen igényváltozások modellezési lehetőségeit.

A következő fejezetben a klasszikus IT infrastruktúra megbízhatóságának, a rendelkezésre állás növelésének lehetőségeit járom körül, majd bemutatom az ún. felhő alapú megoldásokat és azok jellemzőit, szolgáltatásait.

Az utolsó részben a rendszer lehetséges menedzselési típusait mutatom be, részletesebben megvizsgálva a proaktív menedzsment jellemzőit, lehetséges alkalmazásait az oktatástámogató rendszerek üzemeltetése esetén.
 