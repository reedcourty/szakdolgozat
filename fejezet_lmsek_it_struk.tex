\chapter{Tanulásmenedzsment rendszerek IT infrastruktúrája}
\section{A három rétegű architektúra}
A webes LSM-ek általában a három rétegű architektúrát követik. Ez a három réteg a webkiszolgáló, az adatbázis és az alkalmazás réteg. A réteges szerkezetnek köszönhetően ezek a rendszerek jól skálázhatóak, hiszen az egyes rétegekben megtalálható szolgáltatásokat gyakran ilyen funkciókkal alakítják ki.
A rétegeknek nem szükséges fizikailag is külön hardverre kerülni (sőt az alkalmazás- és webkiszolgáló-réteg esetében ez általában nem is lehetséges), így a legegyszerűbb kialakítás akár egy számítógépet is igénybe vehet. Ez a megoldás egy viszonylag erős konfiguráció és alacsony felhasználószám esetén működhet.
\subsection{A webkiszolgáló-réteg}
\todo{Webszerver}
\subsection{Az alkalmazásréteg}
\todo{Alkalmazás szerver}
\subsection{Az adatbázisréteg}
\todo{DB}

És itt elő is kerül az elterjedtebb LMS-ek előnye: a könnyű telepítés, beüzemelés. Vegyük például

\todo{Moodle részletesebben}

\todo{Moodle adodb ábra}
