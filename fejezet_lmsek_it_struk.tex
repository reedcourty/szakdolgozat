\chapter{Tanulásmenedzsment rendszerek IT infrastruktúrája}
\section{A három rétegű architektúra}
A webes LSM-ek általában a három rétegű architektúrát követik. Ez a három réteg a webkiszolgáló, az adatbázis és az alkalmazás réteg. A réteges szerkezetnek köszönhetően ezek a rendszerek jól skálázhatóak, hiszen az egyes rétegekben megtalálható szolgáltatásokat gyakran ilyen funkciókkal alakítják ki.

\begin{figure}[!ht]
\centering
\includegraphics[width=100mm, keepaspectratio]{figures/3tier_simple_001.png}
\caption{A három rétegű architektúra.}
\label{fig:3tier_simple_001}
\end{figure}

\Aref{fig:3tier_simple_001}.~ábrán látható móddal ellentétben a rétegeknek nem szükséges fizikailag is külön hardverre kerülni (sőt az alkalmazás- és webkiszolgáló-réteg esetében ez nem is mindig lehetséges), így a legegyszerűbb kialakítás akár egy számítógépet is igénybe vehet. Ez a megoldás egy viszonylag erős konfiguráció és alacsony felhasználószám esetén működhet.

\subsection{A webkiszolgáló-réteg}
A webkiszolgáló-réteg feladatát egy szolgáltatás látja el, amely futhat egy vagy több példányban, és egy példány futhat egy vagy több számítógépen is. Ez a szolgáltatás felelős azért, hogy a kliensek kérésének megfelelően előálljon a weblap, vagyis a kliensek kérésre kapjanak egy HTML dokumentumot, amelyet a felhasználói oldalon jelenítenek meg.
A piacon több webkiszolgáló alkalmazást találunk, ezek közül van ingyenes, nyílt forráskódú (pl. Apache), és kereskedelmi termék is (Microsoft IIS).

\Aref{fig:netcraft_webservers}.~ábrán a legelterjedtebb webkiszolgálók piaci részesedésének alakulása látható.

\begin{figure}[h!]
\centering
\includegraphics[width=1.0\textwidth]{figures/wpid-overallc.png}
\caption{A legelterjedtebb webszerverek piaci részesedése (2011. november, forrás: \href{http://news.netcraft.com/archives/2011/11/07/november-2011-web-server-survey.html}{Netcraft.com}) \label{fig:netcraft_webservers}}
\end{figure}

Az oktatás támogató rendszerek szempontjából fontos, hogy a működtetett webkiszolgáló képes legyen nagy számú konkurens kérés kiszolgálására, vagy könnyen skálázható, elosztható legyen. LMS-ek esetén is ugyanazokat a webkiszolgálókat lehet használni, mint amit más egyéb webes alkalmazások esetén (pl. Apache, Microsoft IIS, lighttpd, ngnix, stb.).
 
\subsection{Az alkalmazásréteg}
Az alkalmazásréteg feladatát az alkalmazás szerver látja el. Az alkalmazás szerver egy olyan biztosított környezet, ahol alkalmazások futhatnak, és amely környezet szempontjából lényegtelen, hogy mik ezek az alkalmazások és mit csinálnak.\cite{serverside} Az alkalmazás szerver eljárások (programok, rutinok, szkriptek) hatékony végrehajtására dedikált erőforrás.

Az első alkalmazás szerverek megjelenésekor fő feladatuk volt, hogy webes alkalmazások esetén a webszerver által megjeleníthető tartalmat állítsanak elő. Ezen továbblépve napjainkban már nem csak az oldalgenerálás funkciója jelenik meg ebben a rétegben, hanem egyéb szolgáltatásokat is implementálnak, mint például klaszterszervezést, terheléselosztást, hibaátállást. Ezen funkciókkal elérhető, hogy az alkalmazásfejlesztőknek csak az üzleti logikával kelljen foglalkozni.

Az LMS-ek lényegi része ebben a rétegben kerül megvalósításra. Mivel a ténylegesen alkalmazott platformtól függ, hogy milyen alkalmazás szervert használunk, ezért alapvetően erről az oldalról nincs megkötés. Ám érdemes figyelembe venni, hogy a kiválasztott platform mennyire skálázható, milyen teljesítményt nyújt a különböző terhelésekre. A legelterjedtebb LMS-ek fejlesztésére használt platformok itt is, mint más webes rendszereknél, a PHP, Java és .Net.

\subsection{Az adatbázisréteg}
Az adatbázis réteg feladatait az adatbázis szerver látja el. Adatbázis szervernek nevezünk egy dedikált szolgáltatást, ami adatbázisokat tesz elérhető, kezelhetővé. Valójában egy felületet biztosít az alkalmazás rétegben található adatfelhasználó alkalmazás és maguk a felhasználandó adatok között.

Az LMS-es oldaláról igény, hogy a használt adatbázis képes legyen nagy számú konkurens tranzakcióra, és optimálisan tároljon nagy, multimédiás adatokat is. Főleg a költségvetéstől és az üzemeltetéstől függően lehet MySQL, PostgreSQL, Microsoft SQL Server vagy Oracle adatbázisszerver is, de természetesen attól is függ, hogy az LMS melyiket támogatja. 

\section{Egy példa}
\subsection{A Moodle rendszer}
Az e-learning rendszerek közül a legelterjedtebb a Moodle (Modular Object-Oriented Dynamic Learning Environment) (\href{http://moodle.org}{http://moodle.org}) nevű tanulásmenedzsment rendszer. Önálló laboratóriumi munkám során ennek a rendszernek a részletesebb megismerése volt az egyik cél.

A Moodle  egy ingyenes és nyílt forrású LMS vagy VLE (Virtual Learning Environment, Virtuális oktató környezet). 2010 októberében 49952 regisztrált Moodle oldal létezett, amelyek összesen mintegy 37 millió felhasználót szolgáltak ki. A legnagyobb rendszertelepítések közé tartozik a tajvani Ming Chuan Egyetem több, mint 63.000 regisztrált, maximálisan 33.000 bejelentkezett felhasználóval naponta \cite{moodleinstplus}.

Maga a rendszer tervezéséből és implementálásából eredően portábilis, hála a PHP nyelvnek. Módosítás nélkül telepíthető Unix, Linux variánsokra, FreeBSD-re, Windows-ra, Mac OS X-re, NetWare-re és egyéb rendszerekre, amelyek támogatják a PHP-t és az ismertebb adatbázis-kezelő rendszereket.

A Moodle architektúrájában elválik a natív adatbázis-kezelő a felsőbb rétegektől. Közöttük egy ADOdb adatbázis absztrakciós réteg található. Az ADOdb egy különálló projekt \\ (\href{http://adodb.sourceforge.net/}{http://adodb.sourceforge.net/}), amely a legelterjedtebb adatbázis-kezelőket támogatja. \Aref{fig:moodlearch}.~ábrán egy korábbi verzió architektúrája látható.

\begin{figure}[h!]
\centering
\includegraphics[width=0.8\textwidth]{figures/moodlearch.png}
\caption{A Moodle architektúrája \label{fig:moodlearch}}
\end{figure}

Az architekturális felépítésből látható, hogy a Moodle az ADOdb által nyújtott rétegen keresztül egységesen éri el az eltérő adatbázis megvalósításokat anélkül, hogy foglalkoznia kellene az ezekből eredő problémákkal.  

Az interoperabilitás több Moodle képességben megjelenik, ilyenek például:
\begin{sajat_itemize}
\setlength{\itemsep}{0pt}
\item autentikáció LDAP-on, Shibboleth-en keresztül
\item kérdések/kérdéssorok importálása/exportálása több formátumban (pl. XML, XHTML stb.)
\item erőforrások kezelése (SCORM, AICC)
\item integráció más tartalom kezelő rendszerekkel (Drupal, Postnuke).
\end{sajat_itemize}

A Moodle fontos tulajdonsága a modularitás, vagyis a rendszer funkcionalitásának könnyű bővíthetőségek. Ezeket plug-inokkal valósítják meg, amelyek fejlesztését különböző API segítik. Ilyen plug-inok pl. a különböző jelentések (reportok), blokkok, portfóliók, tárolók (repository-k), kereső modulok. Ezek közül nagyon sok megtalálható az alap Moodle telepítésben is, de saját magunk is írhatunk hasonló kiegészítőket.

\subsection{A Moodle egy lehetséges infrastruktúra kialakítása}

Mint azt már említettem, a Moodle nagyon sok operációs rendszerre telepíthető. Én az egyik legelterjedtebb telepítési környezet választottam ki, egy ún. LAMP (Linux - Apache - MySQL - PHP) megoldást használtam. Fizikai szerver helyett virtuális szervert alkalmaztam, a Simonyi Károly Szakkollégium Kollégiumi Számítástechnikai Kör erőforrásainak igénybevételével. A virtualizációs megoldás alapja egy VMware ESX szerver volt.
A szerver konfigurációja viszonylag gyengének számít, ám nem is volt cél, hogy nagy számú felhasználót szolgáljon ki:
\begin{sajat_itemize}
\item Ubuntu Server 10.10 32bit
\item 512 MiB RAM
\item 10 GiB tárhely
\item 4 MiB RAM a videóvezérlőnek 
\end{sajat_itemize}
Az így kialakított struktúra tulajdonképpen egy egy gépes megoldás, mind a webszerver, alkalmazás szerver, mind az adatbázis szerver egy virtuális gépen futott. Természetesen az infrastruktúra méretét lehetett volna növelni, bonyolítani monitorozó, naplófeldolgozó, biztonsági mentés készítő megoldásokkal, de ez szintén nem volt a feladat része.
