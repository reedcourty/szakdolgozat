\chapter{Információs technológiai infrastruktúrák}
\section{A klasszikus IT infrastruktúra}
A klasszikusnak mondott IT infrastruktúrát az írás korábbi részében már bemutattam. Itt most a háromrétegű architektúra egyes rétegeinek skálázását, megbízhatósági szintjének növelésére adott lehetőségeket szeretném bemutatni.
\subsection{Adatbázis réteg}
\todo{Klaszterek, replikák, stb.}
\subsection{Alkalmazás réteg}
\todo{PHP, Java skálázás?}
\subsection{Webkiszolgáló réteg}
\todo{Loadbalancing} 
\section{Felhőalapú infrastruktúrák}
\subsection{Mi is az a ,,számítási felhő''?}

A ,,számítási felhő'' (angolul \foreignlanguage{english}{cloud computing}) egy modell kényelmes, hálózaton keresztül hozzáférhető, konfigurálható számítási erőforrások (pl. hálózat, szerverek, tárhelyek, alkalmazások és szolgáltatások) egy megosztott készletének elérhetőségére, mely erőforrásokat minimális intézkedési erőfeszítéssel vagy szolgáltatói közbenjárással gyorsan rendelkezésre lehet bocsátani \cite{nistsp800-145}.

\todo{Lehet, hogy még finomítani kell a fordításon!}

\begin{comment}
Cloud computing is a model for enabling ubiquitous, convenient, on-demand network access to a shared pool of configurable computing resources (e.g., networks, servers, storage, applications, and services) that can be rapidly provisioned and released with minimal management effort or service provider interaction. 
\end{comment}

Általában hivatkozás szintjén nincsenek elkülönítve az Interneten keresztül szolgáltatott alkalmazások, és a felhő infrastrukturális részét képező hardverek, szoftverek, amelyek ezeket az alkalmazásokat elérhetővé teszik. Ahogy azt \aref{fig:cloud_computing_hu}.~ábrán is szemléltetni próbálom a felhő részét képezi az alkalmazás, szolgáltatás, és a hardver is.

\begin{figure}[h!]
\centering
\includegraphics[width=1.0\textwidth]{figures/Cloud_computing_hu.png}
\caption{A számítási felhő (\foreignlanguage{english}{cloud computing}) (Forrás: \href{https://en.wikipedia.org/wiki/File:Cloud\_computing.svg}{Wikipedia})} \label{fig:cloud_computing_hu}
\end{figure}

\todo{Object Storage ?= Objektum tároló}

\todo{Queue ?= Üzenetsorok}

\todo{Runtime ?= Számításelosztás}

Mélyebb elemzés során azonban a felhőt rétegekre lehet bontani, amely rétegeket a következő alfejezetben részletezném.

\subsection{A felhő rétegei}

\todo{Ezek lehet inkább szolgáltatás modellek?}

A felhő napjainkban négy architekturális rétegből épül fel, amelyeket alulról fölfelé érdemes megvizsgálni.


\begin{figure}[h!]
\centering
\includegraphics[width=0.25\textwidth]{figures/cloud_retegek.png}
\caption{A számítási felhő rétegei \label{fig:cloud_retegek}}
\end{figure}

\begin{figure}[h!]
\centering
\includegraphics[width=0.5\textwidth]{figures/cloud_service_models.png}
\caption{A számítási felhő szolgáltatási modelljei \label{fig:cloud_service_models}}
\end{figure}
 
\subsubsection{Adattár, mint szolgáltatás (\foreignlanguage{english}{data-Storage-as-a-Service, dSaaS})}
Ezt a réteget \todo{szolgáltatást???} nem minden irodalom szokta említeni, ám én itt mégis külön kezelném, hiszen ez a felhő legalapvetőbb szolgáltatása. Lényege, hogy online tárhelyet biztosít a felhasználóknak. Ilyen szolgáltatást nyújt pl. a \href{http://www.dropbox.com}{Dropbox.com} (főleg személyes felhasználásra, biztonsági mentés, megosztás céljából) vagy az \href{https://aws.amazon.com/s3/}{Amazon S3} (inkább nagy szolgáltatók használják).

A dSaaS oktatási rendszerek esetében sok, nagyméretű adatok esetén lehet előnyös, hiszen nem kell a saját szerverünkön tárolni ezeket, megspórolva ezzel saját adattároló rendszer kialakítását, üzemeltetését. Érdemes megjegyezni, hogy sok adatpéldány esetén is érdemes lehet hasonló szolgáltatás használata, hiszen ebben az esetben az autentikációhoz kötött adatoknál már nem kell a session-öket kezelni, az adatok elérhetőségét megadó URL már tartalmaz egy kódot, csökkentve ezzel a web és alkalmazás szerver terheltségét. 

\todo{dSaaS-ról még valami?}

\todo{Ha a két ábrát össze kellene lőni, ez a legalsó réteg.}

\subsubsection{Infrastuktúra, mint szolgálatás (\foreignlanguage{english}{Infrastructure-as-a-Service, IaaS})}

\todo{IaaS}
\subsubsection{Platform, mint szolgáltatás (\foreignlanguage{english}{Platform-as-a-Service, PaaS})}
\todo{PaaS}
\subsubsection{Szoftver, mint szolgáltatás (\foreignlanguage{english}{Software-as-a-Service,SaaS})}
\todo{SaaS}


