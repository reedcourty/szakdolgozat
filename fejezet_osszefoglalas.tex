\chapter{Összefoglalás}

Írásomban megpróbáltam megvizsgálni, hogy az oktatás támogató rendszerek esetében mennyire lehet alkalmazni az általános rendszerek esetén alkalmazott technikákat az erőforrásigények változása szempontjából.

Először bemutattam, hogy mik azok a cél alkalmazások, amelyeket vizsgálok, hol alkalmazzák őket, milyen technológiákra épülnek, majd bemutattam, hogy egy informatikai infrastruktúra esetében mik az oktatás támogató rendszerek által támasztott igények. Leírtam, hogy egy példa rendszer, a Moodle LMS milyen felépítéssel és igényekkel rendelkezik.

A következő fejezetben összegyűjtöttem, hogy az oktatás támogató rendszerek milyen erőforrásigényekkel rendelkeznek, hogyan változhatnak ezek az igények, és mik lehetnek ezen változások okozói. A fejezet végén a lehetséges modelleket próbáltam meg leírni.

Ezután a klasszikus IT infrastruktúra magas rendelkezésre állás elérésének lehetőségeit szedtem össze, majd a fejezet nagy részében foglalkoztam a napjainkban igen elterjedt felhőalapú infrastruktúrával és annak szolgáltatásaival. Igyekeztem az LMS-ek szempontjából vizsgálni az egyes részeket.

Szakdolgozatom utolsó részében arra kerestem a választ, hogy miben különbözik egy reaktív és proaktív infrastruktúra menedzsment, és az oktatás támogató rendszerek üzemeltetését hogyan segítheti a proaktív megoldás.

