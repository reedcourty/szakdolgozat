%----------------------------------------------------------------------------
% Abstract in hungarian
%----------------------------------------------------------------------------
\chapter*{Kivonat}\addcontentsline{toc}{chapter}{Kivonat}

Napjainkban egyre több területen alkalmazzák sikerrel az elektronikus oktatást. Ezen rendszerek üzemeltetése nagyrészt a szokásos -- egyéb rendszerekre is jellemző -- feladatokat adja a kiszolgáló infrastruktúrát felügyelők számára, azonban az erőforrásigényük ingadozása igen speciális. Ezért érdemes áttekinteni, hogy az oktatástámogató rendszerek használatához kapcsolódóan keletkező információk milyen segítséget jelenthetnek az ezeket a rendszereket kiszolgáló infrastruktúra felügyeleténél.

A téma nehézségét adja, hogy nem rendelkezünk az oktatást támogató rendszerek erőforrásigényei és az IT infrastruktúra elérhető erőforrásai közötti megfeleltetésekkel.

Feladatom volt, hogy megismerkedjek ezen oktatástámogató rendszerek használati területeivel, felépítésével, erőforrásigényeivel és azok várható változásaival. Megvizsgáltam a felhőalapú infrastruktúra erőforrás-kezelési lehetőségeit, és megpróbáltam proaktív megoldást adni az oktatástámogató rendszerek erőforrás-kezelésére.
\vfill

%----------------------------------------------------------------------------
% Abstract in english
%----------------------------------------------------------------------------
\chapter*{Abstract}\addcontentsline{toc}{chapter}{Abstract}

Nowadays the e-learning is successfully applied in many field. The operating of these systems mostly provides the usual tasks for the managers of the serving infrastucture, however, there is the fluctuation of the resource needs which is very special. It's worth to reviewing that, how these generated information from the use of learning management systems can help the management of the serving infrastructure.

In this subject the difficulty is that, we do not have matching between the resource needs of an e-learning system and the available resources of the IT infrastructure.

My task was to familiarize myself with the support areas, structure, resource needs of these learning management systems and the changes of these resource needs. I examined the resource management capabilities of cloud-based infrastructure, and tried to find a proactive solution to supporting the e-learning systems of resource management.

\vfill

